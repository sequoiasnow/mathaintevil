\documentclass[10pt]{article}

\usepackage{amsmath}
\usepackage{amsthm}
\usepackage{mathrsfs}
\usepackage{amssymb} 
\usepackage{mathtools}
\usepackage{physics}
\usepackage{tikz-cd}
\usepackage{scalerel}
\usepackage{cancel}

% Set the margins 

\usepackage
[
        a4paper,
        hmargin=4.5cm
]
{geometry}

% Avoid inline equation breaking 

\binoppenalty=\maxdimen
\relpenalty=\maxdimen

% Set the main fonts

\usepackage[lf]{Baskervaldx} % lining figures
\usepackage[bigdelims,vvarbb]{newtxmath} % math italic letters from Nimbus Roman
\usepackage[cal=boondoxo]{mathalfa} % mathcal from STIX, unslanted a bit
%\renewcommand*\oldstylenums[1]{{\textosf{#1}}}

% Create a freespace command

\newcommand{\freespace}[1] {{\F\langle #1 \rangle}}

% Create a vectorspace command 

\newcommand{\vecspace}[1]{{V\langle #1 \rangle}}


% Large plus command

\DeclareMathOperator*{\bigplus}{\scalerel*{+}{\textstyle\sum}}

% Create the power set equation

\newcommand{\powerset}[0] {\textbf{P}}

% Create better title sections

\newcommand{\exercise}[1] {\vspace{0.25in}\large\textit{Exercise #1\quad}}

% Better mapsto and \to command

\renewcommand{\mapsto}[0]{\longmapsto}
\renewcommand{\to}[0]{\longrightarrow}

% Create a category command

\newcommand{\cat}[1]{{\textbf{#1}}}

% Create a better and command 

\newcommand{\writeand}[0]{{\quad \textrm{and} \quad}}

% Create a better iff command 

\newcommand{\writeiff}[0]{{\qquad \textrm{iff} \qquad}}

% Create a category command


% Create a better set command

\DeclarePairedDelimiterX\set[1]\lbrace\rbrace{\def\given{\;\delimsize\vert\;}\,#1\,}

% Define common variables

\newcommand{\N}{{\mathbb{N}}}
\newcommand{\Z}{{\mathbb{Z}}}
\newcommand{\C}{{\mathbb{C}}} 
\newcommand{\R}{{\mathbb{R}}}
\newcommand{\Q}{{\mathbb{Q}}}
\newcommand{\F}{{\mathbb{F}}}


% Better EmptySet

\let\oldemptyset\emptyset
\let\emptyset\varnothing

\linespread{1.3}


\begin{document}

\begin{equation}
G_{im} = 0 \tag{2}
\end{equation}

Diese Gleichungen lassen sich einfacher gestalten, wenn man das Bezugsystem so w\"{a}hlt, da{\ss} $\sqrt{-g} =1$ ist. Dann verschwindet $S_{im}$ wegen (1 b), so da{\ss} man statt erz\"{a}hlt
\begin{align}
R_{im} = \sum_{l} \pdv{\Gamma^{l}_{im}}{x^l} &+ \sum_{\rho l} \Gamma^l_{i\rho} \Gamma^\rho_{ml} = 0 \tag{3} \\
\sqrt{-g} &= 1 \tag{3\;a}
\end{align}

Dabei ist
\begin{equation}
\Gamma^{l}_{im} = - \qty{\begin{array}{cc} im \\ l \end{array}}
\end{equation}
gesetzt, welch Gr\"{o}sen wir als ``Komponenten'' des Gravitationsfeldes bezeichnen.

Ist in dem betrachteten Raume ``Materie'' vorhanden, so tritt deren Energietensor auf den rechten seite von (2) bzw. auf (3) auf. Wir setzen
\begin{equation}
G_{im} = -\kappa\qty(T_{im} - \frac{1}{2}g_{im}T), \tag{2 a}
\end{equation}\\[1cm]

\end{document}
