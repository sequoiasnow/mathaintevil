\documentclass[10pt, oneside]{extarticle}

\usepackage[12pt]{moresize}
\usepackage{amsmath}
\usepackage{amsthm}
\usepackage{mathrsfs}
\usepackage{amssymb} 
\usepackage{mathtools}
\usepackage{braket}
\usepackage{physics}
\usepackage{tikz-cd}
\usepackage{scalerel}
\usepackage[bottom]{footmisc}
\usepackage{cancel}
\usepackage{todonotes}
\usepackage{graphics}
\usepackage{tabularx}
\usepackage[english]{babel}
\usepackage{blindtext}

% Set the title
\def\articletile{Change of Variables Formula}

% Set the margins 

\usepackage
[
        a4paper,
        margin=1.5in
]
{geometry}

% Better norm command

\newtheorem*{theorem}{Theorem}
\newtheorem*{proposition}{Proposition}
\newtheorem*{corollary}{Corollary}
\newtheorem*{lemma}{Lemma}

% Better tables

\newcolumntype{C}{>{$}c<{$}} 
\newcolumntype{L}{>{$}l<{$}} 

% Flip command

\newcommand{\flip}[1]{\reflectbox{\,$#1$\,}}

% Avoid inline equation breaking 

\binoppenalty=999
\relpenalty=999

% Set the main fonts

%\usepackage{fourier}
%\usepackage{courier}
\usepackage{concmath}
\renewcommand*\familydefault{\ttdefault} 
\usepackage[T1]{fontenc}

% Create a freespace command

\newcommand{\freespace}[1] {{\F\langle #1 \rangle}}

% Create a vectorspace command 

\newcommand{\vecspace}[1]{{V\langle #1 \rangle}}

\newcommand{\iproduct}[1]{\langle #1 \rangle}

% The best command

\newcommand{\otherwise}[0]{\text{otherwise}}

% Large plus command

\DeclareMathOperator*{\bigplus}{\scalerel*{+}{\textstyle\sum}}

% Create the power set equation

\newcommand{\powerset}[0] {\textbf{P}}

% Create a new environment a

\newenvironment{question}{\begin{it}}{\end{it}}

% Include a hand drawn graphic into the book.

\newcommand{\includedrawngraphic}[1]{
  \begin{center}
    \vspace{0.4cm}
    \includegraphics[width=\textwidth/2]{#1}
    \vspace{0.4cm}
    \end{center}
  }
  
% Improve the chapter section

\usepackage{titlesec}
\titleformat{\chapter}[display]
{\normalfont\bfseries}
{\HUGE\thechapter}
{1ex}
{\vspace{2ex}%
\LARGE\uppercase}
[\vspace{2in}]

\titleformat{\section}[hang]
{\normalfont\bfseries}
{}
{1ex}
{\large\uppercase}
[]

% Improve proof environment 

\usepackage{changepage}

\renewcommand{\qedsymbol}{\rule{0.7em}{0.7em}}
\renewenvironment{proof}
{\vspace{.5cm}\begin{adjustwidth}{2.7em}{}\noindent{\bfseries Proof \quad}}
{\qed \end{adjustwidth}\vspace{1cm}}


% Better mapsto and \to command

\renewcommand{\mapsto}[0]{\longmapsto}
\renewcommand{\to}[0]{\longrightarrow}

% Create a category command

\newcommand{\cat}[1]{{\textbf{#1}}}

% Create a Jacobian command

\newcommand{\jac}[0]{{\text{Jac}}}

% Create a better and command 

\newcommand{\writeand}[0]{{\quad \textrm{and} \quad}}
\newcommand{\writewhere}[0]{{\qquad \textrm{where} \quad}}

% Create a better iff command 

\newcommand{\writeiff}[0]{{\qquad \textrm{iff} \qquad}}


% Define common variables

\newcommand{\adj}[0]{\text{adj}}
\newcommand{\inv}[0]{\text{Inv}}
\newcommand{\N}{{\mathbb{N}}}
\newcommand{\Z}{{\mathbb{Z}}}
\newcommand{\C}{{\mathbb{C}}} 
\newcommand{\R}{{\mathbb{R}}}
\newcommand{\Q}{{\mathbb{Q}}}
\newcommand{\F}{{\mathbb{F}}}
\newcommand{\A}{{\mathcal{A}}}
\newcommand{\T}{{\mathcal{T}}}
\renewcommand{\L}{{\mathcal{L}}}
\newcommand{\M}{{\mathcal{M}}}
\newcommand{\U}{{\mathcal{U}}}

% Shorthand for exercises
\newcommand{\exercise}[1]{\vspace{1cm}\noindent\textbf{Exercise #1}\quad }

% Matrix shorthand

\newcommand{\m}[1]{\begin{bmatrix} #1 \end{bmatrix}}

% Better EmptySet

\let\oldemptyset\emptyset
\let\emptyset\varnothing

\linespread{1.2}

% Heading

\pagestyle{headings}
\markright{\textnormal{Sequoia Snow}\hfill \small\textnormal{\uppercase{\articletile}} \hfill}


% Define the title separate from the rest of the document

\title{
\textmd{\textbf{\articletile}}\\
}

\author{\textbf{Sequoia Snow}}

% Remove the date

\date{}



\begin{document}

\maketitle

There are few things a ubiquitous in calculus as the $u$-substitution. Its formulation is given as 
\[
  \int_{a}^{b} (f \circ u)(x) \dv{u}{x}  \dd{x} = \int_{u(a)}^{u(b)} f(u) \dd{u}.
\]
This statement is so common in lower level mathematics that it is almost never accompanied by a proof, a matter which we hope to correct. In fact, we will prove a more generalized notion, known as the change of variables formula. 

\begin{theorem}
  Let $U \subset \R^n$ be an open subset, and $\phi : U \to \R^n$ an injective differentiable function with a Jacobean that is never zero. Then for any continous function $f$ which is compactly supported in $\phi(U)$, 
  \[
    \int_{U} f \circ \phi \abs{\jac\phi} = \int_{\phi(U)} f .  \label{eq:cov} \tag{*}
  \] 
\end{theorem}

We will prove this by means of two Lemma's concerning differential forms and stokes theorem. 

\begin{lemma}
  For $\phi : \R^n \to \R^n$ differentiable, and $h$ differentiable with nonzero Jacobian,
  \[
    f^*(h \dd{x_1} \wedge \dots \wedge \dd{x_n}) = \qty(f \circ h) {\jac \phi} \dd{x_1} \wedge \dots \wedge \dd{x_n}.
  \] 
\end{lemma}

\begin{proof}
  The pullback of $f$ has the property that for a function $g$ and a $k$-form $\omega$,
    \[
      f^*(g \cdot \omega) = (g \circ f) \cdot f^* \omega.
    \]
  Thus, as 
  \[
     f^*(h \dd{x_1} \wedge \dots \wedge \dd{x_n}) = \qty(f \circ h) f^*\qty( \dd{x_1} \wedge \dots \wedge \dd{x_n}).
  \]
  it is sufficient to show that 
  \[
    |\jac f|  \dd{x_1} \wedge \dots \wedge \dd{x_n} = f^*\qty( \dd{x_1} \wedge \dots \wedge \dd{x_n}).
  \]
  Let $J_{ij}$ denote the $i,j$th component of the matrix $D\phi$. 
  \begin{align*}
    f^*\qty( \dd{x_1} \wedge \dots \wedge \dd{x_n}) &= \dd{f_1} \wedge \dots \wedge \dd{f_n} \\
    &= \qty( \sum_{j = 1}^n J_{1j} \dd{x_j} ) \wedge \dots \wedge \qty( \sum_{j = 1}^n J_{nj} \dd{x_j} ) \\
    &= \sum_{j_1, j_2, \dots, j_n = 1}^n J_{1j} \cdots J_{nj} \dd{x_{j_1}} \wedge \dots \wedge \dd{x_{j_n}}.
  \end{align*}
  Since all repeated indices cancel out, we are left with all permutations of $\Set{1, \dots, n}$, which we will denote $S_n$. Thus 
  \begin{align*}
   f^*\qty( \dd{x_1} \wedge \dots \wedge \dd{x_n}) 
   &= \sum_{\pi \in S_n} J_{1\pi(1)} \cdots J_{1\pi(n)} \dd{x_{\pi(1)}} \wedge \dots \wedge \dd{x_{\pi(n)}} \\
   &= \det J  \dd{x_1} \wedge \dots \wedge \dd{x_n}.
  \end{align*}
  This follows from the definition of determinant, as $\jac \phi = \det J$ this proves the lemma. 
\end{proof}

\begin{lemma}
  For a differentiable function $\phi : \R^n \to \R^n$, a $k$-form $\omega \in \Omega_k(\R^n)$ and a $k$-cell $c \in C_k(\R^n)$,
  \[
    \int_c \phi^*\omega  = \int_{\phi_* c} \omega.
  \]  
\end{lemma}

\begin{proof}
    This lemma is inherent if we consider differential forms as acting upon functionals, in which case 
    \[
      \qty(\phi^*\omega)(c) = \int_c \phi^* \omega = \omega(\phi_*c) = \int_{\phi_* c} \omega
    \]
    by definition.
\end{proof}

We can now see that the theorem is elementary in this context of differential forms.

\begin{proof}
  By the fist lemma we see that the integrand of the left hand side of \eqref{eq:cov} is equivalent to 
  \[
      f \circ \phi \abs{\jac\phi} \dd{x_1} \wedge \dots \wedge \dd{x_n}
      = \phi^* \qty( f\dd{x_1} \wedge \dots \wedge \dd{x_n}).
  \]
  Now, following the second lemma it is clear that 
  \[
    \int_U  \phi^* \qty( f\dd{x_1} \wedge \dots \wedge \dd{x_n}) = \int_{\phi(U)} f\dd{x_1} \wedge \dots \wedge \dd{x_n}.
  \]
  This completes the proof.
\end{proof}

\end{document}
