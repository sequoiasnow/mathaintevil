\documentclass[10pt]{article}

\usepackage{amsmath}
\usepackage{amsthm}
\usepackage{mathrsfs}
\usepackage{amssymb} 
\usepackage{mathtools}
\usepackage{physics}
\usepackage{tikz-cd}
\usepackage{scalerel}
\usepackage{cancel}

% Set the margins 

\usepackage [
  a4paper,
  hmargin=4.5cm
]{geometry}

% Avoid inline equation breaking 

\binoppenalty=\maxdimen
\relpenalty=\maxdimen

% Set the main fonts

\usepackage[lf]{Baskervaldx} % lining figures
\usepackage[bigdelims,vvarbb]{newtxmath} % math italic letters from Nimbus Roman
\usepackage[cal=boondoxo]{mathalfa} % mathcal from STIX, unslanted a bit
%\renewcommand*\oldstylenums[1]{{\textosf{#1}}}

% Create a freespace command

\newcommand{\freespace}[1] {{\F\langle #1 \rangle}}

% Create a vectorspace command 

\newcommand{\vecspace}[1]{{V\langle #1 \rangle}}


% Large plus command

\DeclareMathOperator*{\bigplus}{\scalerel*{+}{\textstyle\sum}}

% Create the power set equation

\newcommand{\powerset}[0] {\textbf{P}}

% Create better title sections

\newcommand{\exercise}[1] {\vspace{0.25in}\large\textit{Exercise #1\quad}}

% Better mapsto and \to command

\renewcommand{\mapsto}[0]{\longmapsto}
\renewcommand{\to}[0]{\longrightarrow}

% Create a category command

\newcommand{\cat}[1]{{\textbf{#1}}}

% Create a better and command 

\newcommand{\writeand}[0]{{\quad \textrm{and} \quad}}

% Create a better iff command 

\newcommand{\writeiff}[0]{{\qquad \textrm{iff} \qquad}}

% Create a category command


% Create a better set command

\DeclarePairedDelimiterX\set[1]\lbrace\rbrace{\def\given{\;\delimsize\vert\;}\,#1\,}

% Define common variables

\newcommand{\N}{{\mathbb{N}}}
\newcommand{\Z}{{\mathbb{Z}}}
\newcommand{\C}{{\mathbb{C}}} 
\newcommand{\R}{{\mathbb{R}}}
\newcommand{\Q}{{\mathbb{Q}}}
\newcommand{\F}{{\mathbb{F}}}


% Better EmptySet

\let\oldemptyset\emptyset
\let\emptyset\varnothing

\linespread{1.3}


\begin{document}

One of the most common problems found in quantum mechanics textbooks, is that of the infinite square well. While it is a very abstracted model, the infinite square well illustrates many of the principles of quantum mechanics and can actually serve as a useful model in cases of very steep potentials. 

Let's get to the actual problem. Consider a particle of mass $m$ and energy $E > 0$ trapped in a potential $V(x)$ as follows.
\[
  V(x) = \begin{cases}
    0  & 0 \leq x \leq L \\
    \infty  & \text{otherwise}  
 \end{cases} 
\]
Now, while this problem is phrased in the context of Physics it reduces to a simple differential equation. Namely, our particle inside the box will satisfy the time independent Schr\"{o}dinger equation 
\[
  \left[ -\frac{\hbar^2}{2m} \frac{d^2}{dx^2} + V(x) \right] \psi = E \psi  .
\]
In our case we know the particle can not exist outside of the box for the purely reason that a particle shouldn't exist in an infinite potential. However, mathematically it is equally clear that as $V \to \infty$ 
\[
  \frac{2m(V - E)}{\hbar^2} \psi = \psi''
\]
can only be satisfied by $\psi = 0$. So for $x \notin [(0, l)]$, $\psi(x) = 0$. What about inside the square well? We can start by solving the differential equation as follow, first let's simplify some of the constants.
\[
  \psi'' = -k^2\psi, \qquad k = \frac{\sqrt{2mE}}{\hbar}
\]
Now as $\psi'' = \alpha \psi$ we know we are looking at an equation of the form $e^{\sqrt{\alpha} x}$. In our case, that is 
\[
  \psi(x) = A e^{ikx} + Bx + C.
\]
This is the general solution to our differential equation. In order for our solution to satisfy the physical constraints of the wave equation it must be continuous, thus $\psi(0) = \psi(L) = 0$. Applying that constraint we immediately see that $C = 0$ and  $B = 0$. Furthermore, we can use Euler's formula to expand $e^{ikx}$ to see that 
\[
  \psi(x) = A\big[ \cos(kx) + i\sin(kx) \big].
\]
Now note that $A$ is a complex variable so we can express it as $A = \alpha -i\beta$, for real numbers $\alpha, \beta \in \mathbb{R}$. With that simplification in hand we can write 
\[
  \psi(x) = \alpha \cos(kx) + \beta\sin(kx).
\] 
Now applying our boundary conditions we see that as $\cos(0) = 1$ the $\cos$ term must disappear.
\[
  \psi(0) = \alpha = 0. 
\] 
The case where $x = L$ is more interesting, for a continuous wave function $\psi$ must satisfy 
\[
  \psi(L) = 0 = \beta \sin(kL).
\]
This is true only when 
\[
    kL = n\pi, \qquad n = 0,1,2,3\dots 
\]
Thus $k$ is \textbf{quantized} by $n$, and since $k$ is determined by the energy of the particle, this implies that only certain energies can exist!
\[
  \frac{\sqrt{2mE_n}}{\hbar}L = n \pi \implies E_n = \frac{n^2\pi^2\hbar^2}{2mL}, \qquad 
  n = 0,1,2,3, \dots 
\]
This quantization is one of the most fundamental results of quantum mechanics, stating that only certain energy levels can exist. This simple model gives quantization as predicted by the Bohr model of the atom. 

Of course, we aren't done yet. To get the full solution we must apply the conditions of normalization, namely 
\[
  \int_{\infty}^\infty |\psi(x)|^2 \dd{x} = 1.
\]
In our case that simplifies to the following formula for $\beta$,
\[
  \int_{-\infty}^\infty |\psi(x)|^2 \dd{x} 
  = \beta^2 \int_{0}^L \sin^2(\frac{n\pi x}{L}) \dd{x} 
  = \beta^2 \left [ \frac{x}{2} - \frac{L\sin(\frac{2n\pi x}{L})}{3n\pi} \Bigg|_0^L \right]
  = \frac{L\beta^2}{2} = 1.
\]
Which implies that $\beta = \sqrt{2/L}$. Thus our energy eigenfunctions are given as 
\[
  \psi_n(x) = \sqrt{\frac{2}{L}}\sin(\frac{n\pi x}{L}).
\]
One of the things that can be confusing about this solution is the arbitrary nature of the coordinates. Suppose instead of picking an interval $[0, L]$ we used the interval $[-\frac{L}{2}, \frac{L}{2}]$? One of the cornerstones of physics is that the natural laws are independent of reference frame, so theoretically the solution to that formula should be identical. However, if you solve it explicitly you end up with a $\sin$ \textbf{and} a $\cos$ term. 

While this may look different, this is in fact identical to our solution under the transformation $x' = x - \frac{L}{2}$. Note that 
\[
  \psi_n(x') = \sqrt{\frac{2}{L}} \sin(\frac{n \pi x}{L} - \frac{n \pi}{2}).
\] 
Recall that that $\sin(x - \frac{\pi}{2}) = \cos(x)$, thus the previous equation gives us $\sin$ or $\cos$ depending on $n$.
\[
  \psi_n(x') = \begin{cases}
    \frac{2}{L} \sin(\frac{n\pi x}{L}) & n = 0, 2, 4, \dots \\
    \frac{2}{L} \cos(\frac{n\pi x}{L}) & n = 1, 3, 5, \dots \\
  \end{cases}
\] 
While this is a minor point its an important one. Often students first exposed to quantum mechanics will be confused by the variety of solutions to a given problem, when in fact they are simply the same solution in different coordinate systems.
\end{document}
