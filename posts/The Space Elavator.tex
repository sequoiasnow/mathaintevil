\documentclass[12pt]{article}

\usepackage{amsmath}
\usepackage{amsthm}
\usepackage{mathrsfs}
\usepackage{amssymb} 
\usepackage{mathtools}
\usepackage{physics}
\usepackage{tikz-cd}
\usepackage{scalerel}
\usepackage{cancel}
\usepackage{todonotes}
\usepackage{tabularx}

% Set the margins 

\usepackage
[
        a4paper,
        hmargin=4.5cm
]
{geometry}

\newcolumntype{C}{>{$}c<{$}} 
\newcolumntype{L}{>{$}l<{$}} 

% Avoid inline equation breaking 

\binoppenalty=\maxdimen
\relpenalty=\maxdimen

% Set the main fonts

\usepackage[lf]{Baskervaldx} % lining figures
\usepackage[bigdelims,vvarbb]{newtxmath} % math italic letters from Nimbus Roman
\usepackage[cal=boondoxo]{mathalfa} % mathcal from STIX, unslanted a bit
%\renewcommand*\oldstylenums[1]{{\textosf{#1}}}

% Create a freespace command

\newcommand{\freespace}[1] {{\F\langle #1 \rangle}}

% Create a vectorspace command 

\newcommand{\vecspace}[1]{{V\langle #1 \rangle}}


% Large plus command

\DeclareMathOperator*{\bigplus}{\scalerel*{+}{\textstyle\sum}}

% Create the power set equation

\newcommand{\powerset}[0] {\textbf{P}}

% Create better title sections

\newcommand{\exercise}[1] {\vspace{0.25in}\large\textit{Exercise #1\quad}}

% Better mapsto and \to command

\renewcommand{\mapsto}[0]{\longmapsto}
\renewcommand{\to}[0]{\longrightarrow}

% Create a category command

\newcommand{\cat}[1]{{\textbf{#1}}}

% Create a better and command 

\newcommand{\writeand}[0]{{\quad \textrm{and} \quad}}

% Create a better iff command 

\newcommand{\writeiff}[0]{{\qquad \textrm{iff} \qquad}}

% Create a category command


% Create a better set command

\DeclarePairedDelimiterX\set[1]\lbrace\rbrace{\def\given{\;\delimsize\vert\;}\,#1\,}

% Define common variables

\newcommand{\N}{{\mathbb{N}}}
\newcommand{\Z}{{\mathbb{Z}}}
\newcommand{\C}{{\mathbb{C}}} 
\newcommand{\R}{{\mathbb{R}}}
\newcommand{\Q}{{\mathbb{Q}}}
\newcommand{\F}{{\mathbb{F}}}


% Better EmptySet

\let\oldemptyset\emptyset
\let\emptyset\varnothing

\linespread{1.3}

\begin{document}

The space elevator is one of the most exiting topics in the future of humanities exploration of space. Not only would it be a feat of engineering that would set a marker for all time of humanities conquest of our solar system, it would make that conquest remarkably cheap and feasible. The general concept of a space elevator is a giant orbiting teether to the earth where spaceships, satellites and astronauts could be lifted by simple electrical power into orbit, and from there be released into space without the requirements of lifting massive amounts of fuel. 

\begin{figure}[htbp]
  \centering
  \includegraphics{figures/the-space-elavator/structural-diagram}
\end{figure}

Mathematically this can be described in terms of Newtonian mechanics. Let $\rho$ be the density of our rope in units of mass/distance. Let $R$ be the distance from the mass $M$ to the center of the earth and $R_E$ the radius of the earth and $\omega$ the Earth's rate of rotation. Now the upwards (centripetal) force of the mass $M$ is given as 
\[
  F_C = \frac{Mv^2}{R}.
\]  
It follows that this must be equal to the gravitational pull between the Eath and the mass \textbf{and} the weight of the teether. The first is given simply as 
\[
  F_E = G \frac{M_E M}{R^2}.
\]
Now the teether is more complicated, it will weigh progressively less as its distance from the earth increases. The total weight of the tether which we will denote $T$ is given as 
\[
  T = \int_{R_E}^R G \frac{M \rho \dd{r}}{r^2}.
\]
However, this isn't quite accurate, recall that there is also a centripetal force pushing away from earth, thus we must add the centripital term, here we phrase this in terms of $v_t(r)$ which is the velocity of each point on the teether as a function of its distance from the center of the earth. 
\[
 T = \int_{R_E}^R G \frac{M \rho \dd{r}}{r^2} - \int_{R_E}^R \frac{v_t(r)^2\rho \dd{r}}{r}.
\]
Thus we are left with the equation 
\[
  \frac{Mv^2}{R} =  \frac{M_E M}{R^2} + \int_{R_E}^R \qty[ \frac{GM}{r^2} - \frac{v_t(r)^2}{r}] \rho \dd{r}.
\]
Obviously this equation has a curve of solutions given various $v, \rho, M, v_t$, however a further constraint is that our rope shouldn't wrap around the earth at any given moment, in fact it should be constant with respect to the rotation about the equator. Thus 
\[
  \omega = \frac{r}{v}
\]
should be constant. This gives us an equation for $v_t(r)$, namely 
\[
  v_t(r) = \frac{r}{\omega}.
\] 
If we substitute this into the previous equation, the number of dynamic variables shrinks significantly, 
\[
  \frac{MR}{\omega} =  \frac{M_E M}{R^2} + \int_{R_E}^R \qty[ \frac{GM}{r^2} - \frac{r}{\omega}] \rho \dd{r}.
\]
If we solve this integral we get that 
\[
  \frac{MR}{\omega} =  \frac{M_E M}{R^2} - \grac{GM\rho}{R^2}
\]


\end{document}
